\documentclass[a4paper,titlepage,8.5pt]{article}
\usepackage{fontspec}
\setmainfont[Ligatures=TeX]{Open Sans}
\usepackage[english]{babel}
\usepackage{csquotes}
\usepackage[titles]{tocloft}
\usepackage{hyperref}
\usepackage[parfill]{parskip}
\usepackage{a4wide}
%\usepackage{draftwatermark}

\title{Pirate Party Australia Constitution}

\hypersetup{
	pdftitle={Pirate Party Australia Constitution},
	pdfauthor={},
	pdfsubject={},
	pdfcreator={},
	pdfproducer={},
	pdfkeywords={},
	colorlinks=false % set to true to see links in red
}

% Set section indent
\setlength{\cftsecnumwidth}{2.0em}
\setcounter{secnumdepth}{5}

\renewcommand{\theenumi}{\arabic{enumi}}
\renewcommand{\labelenumi}{(\theenumi)}

\renewcommand{\theenumii}{\alph{enumii}}
\renewcommand{\labelenumii}{(\theenumii)}

\renewcommand{\theenumiii}{\roman{enumiii}}
\renewcommand{\labelenumiii}{(\theenumiii)}

\renewcommand{\theenumiv}{\roman{enumiv}}
\renewcommand{\labelenumiv}{(\theenumiv)}

\begin{document}

\tableofcontents
\thispagestyle{empty}
\newpage

% Reset page numbers so the TOC doesnt count...
\setcounter{page}{1}

\part{Principles \& Objects of the Party}
Pirate Party Australia strives to protect and expand civil and digital liberties, social equality and freedom of culture. We seek to safeguard the inalienable rights of all natural persons of Australia and the freedoms of the emergent global information society. The Party seeks to have these values reflected in the laws and institutions of Australia.

The growing surveillance of the citizen offends the very notions of a liberal and open democracy. Overbearing and restrictive private monopolies constructed via regimes of antiquated, unfair and unbalanced laws which prevent the free development of culture and ideas are detrimental to financial, economic and cultural outcomes for the citizens of Australia. Changing these laws, and ensuring the protection of these values are the goals of Pirate Party Australia.

Founded on the same principles as other international Pirate Parties, it is part of a global movement against increasingly draconian copyright and patent laws, and the erosion of the right to privacy. The basic tenets of this movement are free culture, civil liberty and intellectual rights reform.

The Party seeks to represent the emerging digital society and offer a new form of politics driven from the grassroots. We believe in exploring the potential of participatory and deliberative democracy, and finding new ways to promote trust between citizens and the state through greater transparency, evidence-based policy, and deliberative and open government. The Party intends to contest Australian Federal Elections in both the House of Representatives and Senate.

An elected representative of the party must not vote for or compromise on any legislation that impinges on or compromises the rights stated here in this Constitution.


Objectives of Pirate Party Australia also include:
\begin{itemize}
\item To construct, advocate and implement policies in accordance with the principles stated within this Constitution; and
\item To generally educate and bring awareness to the issues that are stated within this Constitution; and
\item To educate and encourage other political entities to adopt our objectives, whether that be through advocacy or preference allocation.
\end{itemize}


Pirate Party Australia firmly holds belief in democracy, and rejects any use of force, intimidation or physical violence as the means to achieving political goals. We vehemently reject any and all forms of political or public corruption.
\newpage

\part{Definitions}

\begin{description}
\item[Simple majority:] One half (1/2), ignoring the remainder, plus one (1) of votes on the motion must be in favour of the motion for it to carry. Abstaining is not considered for the purposes of calculating the majority but still contributes to meeting the relevant quorum.
\item[Absolute majority:] One half (1/2), ignoring the remainder, plus one (1) of all members who have the right to vote on the motion must vote in favour of the motion for it to carry. For the purposes of this type of majority, abstaining is equivalent to voting against the motion.
\item[Two-thirds majority:] Two-thirds (2/3), ignoring the remainder, plus one (1) of votes on the motion must be in favour of the motion for it to carry. Abstaining is not considered for the purposes of calculating the majority but still contributes to meeting the relevant quorum.
\item[Absolute two-thirds majority:] Two-thirds (2/3), ignoring the remainder, plus one (1) of all members who have the right to vote on the motion must vote in favour of the motion for it to carry. For the purposes of this type of majority, abstaining is equivalent to voting against the motion.
\item[Three-quarters majority:] Three-quarters (3/4), ignoring the remainder, plus one (1) of votes on the motion must be in favour of the motion for it to carry. Abstaining is not considered for the purposes of calculating the majority but still contributes to meeting the relevant quorum.
\item[Optional preferential voting:] A type of voting where the voter may opt to fill in as few as none and as many as all of the fields provided, with numbering relevant to the voting system being used.
\item[Schulze voting method:] A Condorcet preferential voting system that compares each candidate by their rank to each other to find the winning candidate.
\end{description}

\part{Articles of the Constitution}

\section{Name, Principles and Constitution}

\begin{enumerate}
\item The name of the party will be ``Pirate Party Australia'', also known as ``the Pirate Party''. Pirate Party Australia may be referred to as ``PPAU'' internally. From hereinafter in this document, Pirate Party Australia shall be referred to as either ``Pirate Party Australia'' or ``the Party''. The principles and objects of Pirate Party Australia are stated in Part I, and are fundamental to the purpose of the Party. All party documents, members and policies are subject and subordinate to this Constitution.
\item Pirate Party Australia is a non-profit organisation. The assets and income of the organisation shall be applied solely in furtherance of its above-mentioned objects and no portion shall be distributed directly or indirectly to the members of the organisation except as bona fide compensation for services rendered or expenses incurred on behalf of the organisation.
\item Pirate Party Australia's financial year shall begin on 1 July and end on 30 June the following year.
\end{enumerate}

\section{Structure \& Composition}

\begin{enumerate}
\item The Party shall be governed at a Federal level by a body entitled the ``National Council''. The National Council may create additional structures and subordinate organisations, such as committees, working groups or branches, as it sees fit.
\item The National Council shall be comprised of those persons formally elected to positions elaborated on within this Constitution.
\item Those members who form the National Council are to be elected from those eligible persons as elaborated within this Constitution, and are to be elected after deliberation at an annual National Congress.
\item The elected members of the National Council shall appoint to the National Council a Party Agent in accordance with Article 3 and Article 10. The Party Agent will fulfil the obligations and duties of the role as provided by the \textit{Commonwealth Electoral Act 1918}.
\item The National Council, as the paramount governing body of the Party, has the authority to overrule or amend any policy or decision of any subordinate organisation (except the Dispute Resolution Committee), if it deems those things to be inconsistent with or repugnant to the values, ideals or policies of the Party.
\item A two-thirds majority vote of the National Council is required for any action from paragraph (5) to be taken against a subordinate organisation or their decisions.
\end{enumerate}

\subsection{State and Territory Branches}

\begin{enumerate}
\item A State or Territory Branch is considered a subordinate organisation of the Party for the purposes of this Constitution.
\item These branches exist for the purpose of contesting state/territory elections, and relevant local elections of their region.
\item The National Council or National Congress may, at its discretion, opt to offer a mandate to form a state or territory branch to the State/Territory Coordinator of the given state/territory.
\item A State or Territory Branch may not contradict a Federal policy, but may expand their policy set beyond the scope of the Federal Party.
\item It is mandatory that members of the Federal Party are members of their local State or Territory Branch, and vice versa.
\item No State or Territory Branch may enact or enforce policies that contradict this Constitution.
\item No State or Territory Branch may register as a federal political party, nor as a division of the Federal Party.
\end{enumerate}

\section{National Council}

\begin{enumerate}
\item Members of the National Council shall be referred to as Councillors.
\end{enumerate}

\subsection{Quorum and Majorities}

\begin{enumerate}
\item Unless otherwise provided within this Constitution, no question regarding Party business is to be decided or resolved at a meeting of the National Council unless at least five (5) members or two-thirds of the National Council are present, whichever number is greater.
\item A Councillor may add their contribution to quorum if they are unable to attend, but only for specified issues, by express, written consent, conditional on the following:
\begin{enumerate}
\item The vote may only be applied where the exact motion text was known to the Councillor in advance, and the vote is for the unmodified text.
\item Written consent should be included within the minutes.
\end{enumerate}
\item All motions must be carried by an absolute two-thirds majority of the National Council.
\item The quorum for any motion to accept the minutes of a previous meeting is set at the minimum to achieve a two-thirds majority of those present. All Councillors absent from the previous meeting abstain by default.
\end{enumerate}


\subsection{Positions}

\subsubsection{President}

\paragraph{Duties and Responsibilities}

\begin{enumerate}
\item Lead the Party.
\item Chair the National Congress, and meetings of the National Council.
\item Co-ordinate the activities of the National Council.
\end{enumerate}

\subsubsection{Deputy President}

\paragraph{Duties and Responsibilities}

\begin{enumerate}
\item Assist the President with their duties in accordance with this Constitution.
\item If the President is unable (on a temporary basis) to conduct their obligations under the Constitution, the Deputy is to substitute and fulfil those obligations.
\end{enumerate}

\subsubsection{Party Secretary}

\paragraph{Duties and Responsibilities}

\begin{enumerate}
\item The Party Secretary fulfils the requirements and obligations of the position of the same name defined in the \textit{Commonwealth Electoral Act 1918}.
\item Provide notice in advance to members of all official meetings, and of the National Congress.
\item Prepare schedules, agenda, and correspondence from members for submission to the meeting or National Congress, and record attendance of persons present, and arrange for minutes or logs to be recorded.
\item Co-ordinate official correspondence of the National Council.
\item Maintain the party register, in accordance with \textit{Commonwealth Electoral Act 1918}.
\item Maintain custody of all documents, statements and records of the Party, and except for those documents that are otherwise accounted for in this Constitution, by other officers.
\item Briefly minute, or delegate responsibility for minuting, listing the decisions of meetings of the National Congress and National Council and ensure publication at the earliest possible convenience.
\end{enumerate}

\subsubsection{Deputy Party Secretary}

\paragraph{Duties and Responsibilities}

\begin{enumerate}
\item Assist the Party Secretary with their duties in accordance with this Constitution.
\item If the Party Secretary is unable (on a temporary basis) to conduct their obligations under the Constitution, the Deputy is to substitute and fulfil those obligations.
\end{enumerate}

\subsubsection{Treasurer}

\paragraph{Duties and Responsibilities}

\begin{enumerate}
\item The receipt of all monies paid to the Party, the issuing of all receipts and the deposit of such monies into accounts determined by the National Council.
\item Develop and ensure security and accountability measures for all receipts and payments are followed.
\item Submit an Annual Financial Report to the National Congress, detailing balance sheets, financial statements and relevant particulars.
\item Maintain adequate controls over Party finances and all financial records, documents, securities ensuring smooth transition when position is transferred.
\item Ensure that all book keeping is conducted by an appropriately skilled person, and all documents conform to relevant legislation and regulations and this Constitution.
\end{enumerate}

\subsubsection{Deputy Treasurer}

\paragraph{Duties and Responsibilities}

\begin{enumerate}
\item Assist the Treasurer with their duties in accordance with this Constitution.
\item If the Treasurer is unable (on a temporary basis) to conduct their obligations under the Constitution, the Deputy is to substitute and fulfil those obligations.
\end{enumerate}

\subsubsection{Registered Officer}

\paragraph{Duties and Responsibilities}

\begin{enumerate}
\item The Registered Officer fulfils the requirements and obligations of the position of the same name defined in the \textit{Commonwealth Electoral Act 1918}.
\item The Registered Officer will fulfil the requirements and obligations of the Party Agent as defined in the \textit{Commonwealth Electoral Act 1918}.
\end{enumerate}

\subsubsection{Councillors}

\begin{enumerate}
\item Two (2) Councillors will be appointed by election at the National Congress to the National Council.
\end{enumerate}

\paragraph{Duties and Responsibilities}

\begin{enumerate}
\item The responsibilities of the Councillors will be determined by the National Council.
\end{enumerate}

\section{Membership}

\subsection{Eligibility}

\begin{enumerate}
\item Membership is open to all natural persons who:
\begin{enumerate}
\item Have read and agreed to the terms and principles provided within this Constitution;
\item Pay an annual membership fee, if applicable, as set by the National Council;
\item Are not currently members of any other registered or unregistered political party in Australia and do not join another party in Australia while a member of the Party;
\item Have not been members of another registered or unregistered political party in Australia in the previous twelve (12) months, unless this is disclosed in the membership application by the applicant; and
\item Are registered on the Australian electoral roll.
\end{enumerate}
\item A Member’s Party membership will not lapse unless the Member resigns from the Party in writing to the Secretariat, or an applicable membership fee is failed to be paid more than ninety (90) days after their membership period has expired.
\item A Member must be sent an email to inform them that their membership will lapse in 30 days before that Member's membership may lapse.
\item The National Council may at its discretion opt to waive membership fees on a case-by-case basis.
\end{enumerate}

\subsection{Categories of Membership}

\subsubsection{Full Membership}

\begin{enumerate}
\item Full Members are entitled to:
\begin{enumerate}
\item Be elected into a formal position within the party, at any level;
\item Where eligible, and approved by the nomination processes within this Constitution, stand as a candidate in any election the party contests;
\item Communicate and submit policy amendment proposals and Constitutional amendment proposals;
\item Participate in policy and issue discussion, debate and partake in the decision making process in accordance with this Constitution;
\item Where eligible, participate in working groups defined by the National Council or any organ of the Party; and
\item Vote at Party Meetings, Congresses and Policy Formulation, Development and Adoption proceedings.
\end{enumerate}
\end{enumerate}

\subsubsection{Associate Membership}

\begin{enumerate}
\item Associate Members are entitled to the same privileges as Full Members, except they:
\begin{enumerate}
\item Are ineligible for National Council and Dispute Resolution Committee positions;
\item May not lead any committee;
\item Do not have voting rights, but have the ability to motion through the sponsorship of a Full Member; and
\item Are not eligible to stand as a candidate in any election the Party contests.
\end{enumerate}
\end{enumerate}

\subsection{Additional Categories of Membership}

\begin{enumerate}
\item The National Council may propose to existing members the creation of additional categories of Membership of the Party. A quorum of 10\% of the entire existing membership is necessary for such a vote, which will be decided by simple majority of those members.
\end{enumerate}

\subsection{Recruitment of Members}

\begin{enumerate}
\item The National Council may specify rules to prevent the abusive recruitment of new members into the Party or abusive renewal of memberships.
\end{enumerate}

\subsection{Refusal, Suspension and Expulsion}

\begin{enumerate}
\item The National Council may refuse to accept an application for membership by any individual on the grounds that the acceptance of the membership may be prejudicial to the interests or values of the Party.
\item The National Council also has the power to suspend or expel a member should that individual’s membership or actions whilst a member be prejudicial to the interests of the Party.
\item Any refusal to admit a person as a member, and any suspension or expulsion of a member, shall be accompanied by a written statement of reasons for the action, and this statement is to be made available to all membership (unless requested to be kept confidential by the member affected).
\item A refusal of an application for membership, or the suspension or expulsion of any member may only be decided by a two-thirds majority vote of the National Council.
\end{enumerate}


\section{Policy Formulation, Development and Adoption}

\subsection{Development}

\begin{enumerate}
\item Policy development must occur with as much interaction with members as is feasible, the process must be as participatory as is feasible, and outcomes must be reached through consensus where feasible.
\end{enumerate}

\subsection{Adoption}

\begin{enumerate}
\item New policy, unless dictated by circumstances of urgency, shall be decided on at the National Congress.
\item Where circumstance of urgency are apparent and declared, the National Council may make policy, that shall be considered official, however that policy is subject to vote at the next National Congress, and is subject to the same conditions as those above.
\item A policy will not be adopted if it is inconsistent with Part I of this Constitution.
\item All policies adopted by the National Congress will be recorded in a central register available to all Members.
\item Every feasible effort must be taken to ensure that there is some accessible and equitable mechanism available for remote participation at the National Congress.
\end{enumerate}

\subsection{Policy Review}

\begin{enumerate}
\item Where not less than 15\% of Full Members petition the National Council, a policy will come under official review by the party, where that policy will be reviewed and voted upon at the National Congress.
\end{enumerate}

\subsection{Positions on issues outside of Platform}

\begin{enumerate}
\item No member of the Party may imply that a personal position on issues outside of the scope of the Party principles, platform or policies is the position of the Party.
\end{enumerate}

\section{Meeting Procedure and Requirements}

\begin{enumerate}
\item Meetings should be structured so as to allow all members to participate, and have their opinions acknowledged.
\item All members should be notified at least 24 hours in advance of any official meeting of the National Council, and of the intended agenda of such meetings.
\item Consensus should be the focus of any proposal or decision at a meeting. However, where consensus cannot be achieved, a two-thirds majority will be sufficient to carry forward a proposal.
\item Where there is disagreement, or members indicate that a delay in voting is required, sufficient time should be given for discussion before any voting begins.
\item Meetings are only open to members unless a simple majority of the members present permit specified non-members to observe the meeting.
\item The method of voting and the medium by which the meeting occurs is to be determined by the meeting facilitator, except where otherwise provided for by this Constitution.
\item The minutes of a meeting should be distributed to the Members within fourteen days of the meeting or before the group next convenes, whichever is shorter. The National Council may specify procedures for the collection and dissemination of such minutes.
\item The National Council may specify additional meetings procedures.
\end{enumerate}

\subsection{National Congress}

\begin{enumerate}
\item The National Council will organise the National Congress.
\item A National Congress must begin in July each year, and shall be referred to as the Annual National Congress where disambiguation is necessary.
\item The National Congress must be announced forty-two (42) days prior to the date of the Congress.
\item The agenda must be finalised at least seven (7) days prior to the date of the Congress.
\item If at least 25\% of the members petition the National Council in writing expressing their lack of confidence in the National Council, the National Council shall organise an emergency National Congress of the Members within thirty (30) days.
\item The National Congress has the exclusive right, by two-thirds majority vote, to allow the Party to merge with, affiliate with or disaffiliate with any other organisation.
\end{enumerate}

\subsection{Preselection Meeting}

\begin{enumerate}
\item The National Council will organise the Preselection Meeting.
\item The Preselection Meeting may be an independent meeting, or may coincide with the National Congress or another meeting.
\item Multiple Preselection Meetings may be held where deemed appropriate by the National Council in the lead up to an election.
\item The National Council may determine that a separate Preselection Meeting may be held for each specific geographic area.
\end{enumerate}

\subsection{Policy Meeting}

\begin{enumerate}
\item The National Council will organise the Policy Meeting.
\item The Policy Meeting may be an independent meeting, or may coincide with the National Congress or another meeting.
\item The Policy Meeting is always considered to coincide with the National Congress.
\item The Policy Meeting may be held as often as deemed appropriate by the National Council.
\end{enumerate}

\subsection{Online Voting}

\begin{enumerate}
\item Some elements of the National Congress and Policy Meetings are required to be put to a final vote on an online voting system.
\item The online voting period must not be less than seven (7) days.
\item The online voting system must ensure that only Full Members can vote, and that each member may only vote once per poll.
\item Motions of the following types that carry at a National Congress will be put to a final vote on an online voting system for Full Members, where said motions will only carry if they pass by the threshold provided for by the Constitution, or where not provided, a two-thirds majority:
\begin{enumerate}
\item Constitutional amendments,
\item Platform amendments, policy amendments and position statements,
\item Other documentation that guides party position or direction, and
\item Party mergers, formal affiliations or disaffiliations with other organisations.
\end{enumerate}
\item Motions of the following types that carry at a Policy Meeting will be put to a final vote on an online voting system for Full Members, where said motions will only carry if they pass by the threshold provided for by the Constitution, or where not provided, a two-thirds majority:
\begin{enumerate}
\item Platform amendments, policy amendments and position statements
\end{enumerate}
\item Officer election requirements as provided for by the Constitution, including for the Dispute Resolution Committee, will be fulfilled by the online voting system.
\end{enumerate}

\section{Pre-Selection of Candidates for Election to Federal Parliament}

\begin{enumerate}
\item All Members seeking to stand as candidates for election to Federal Parliament must be nominated at a Preselection Meeting and seconded by another member.
\item The National Council will determine whether all members (or a geographical sub-set of members) will vote to select candidates for election to Federal Parliament.
\item All members seeking to stand as candidates must submit to the National Congress a detailed and truthful statement as to their suitability for election.
\item The National Council may establish procedures for the vetting of candidates backgrounds and must publish these procedures to the Membership.
\item As far as is practicable, candidates should be selected at least twelve (12) months before the normal time of the next election.
\item All members wishing to run as candidates for Pirate Party Australia must sign a declaration to the effect of:
\begin{enumerate}
	\item I hereby pledge to advance and adhere to the platform and ideals of Pirate Party Australia, both during the election campaign and upon election to Parliament.
\end{enumerate}
\end{enumerate}

\section{Financial Structure}

\subsection{Property}

\begin{enumerate}
\item All property and resources of the Party are to be used solely for the purposes of promoting and achieving the principles and goals stated within this Constitution.
\item All Members, upon request to the National Council, may have access to the latest financial reports of the Party.
\item All bank accounts of the Party will:
\begin{enumerate}
\item be held separately from those of its members;
\item require more than one signatory for the disbursement of funds; and
\item include the wording Pirate Party in their title.
\end{enumerate}
\item All non-banking financial accounts (for example, PayPal) of the Party will:
\begin{enumerate}
\item be held separately from those of its members;
\item have multiple signatories/user accounts linked to the account, if possible;
\item move all funds into the bank accounts as soon as feasible; and
\item have all records published annually.
\end{enumerate}
\item All accounts of the Party will be audited annually and the auditor’s report published to the members at the National Congress.
\end{enumerate}

\section{Constitutional Amendments, Interpretation and By-Laws}

\subsection{Amendments}

\begin{enumerate}
\item The Constitution may only be amended during the National Congress. Amendments require a two-thirds majority vote with a quorum of twenty (20) percent of Members at the time the amendment was proposed.
\item Constitutional amendment proposals must be submitted by email to the Secretary by 9:00 am AEST on the Saturday that falls on the 28th day before the first day of the National Congress.
\item Members must be notified by email of any proposals for amendments by 11:59 pm AEST on the Saturday that falls on the 28th day before the first day of the National Congress. This requirement can be fulfilled by placing the proposals at a specified place on the Party website or wiki before this deadline and informing the membership of their location.
\item New proposals may not be added after the deadline specified in Article 9.1(2), but already proposed amendments may be modified by the proposer prior to the National Congress, so long as:
\begin{enumerate}
  \item These modifications do not substantially change the proposed amendments but may include updates in wording for clarity, or to correct errors; and
  \item These modifications are recorded and justified.
\end{enumerate}

\end{enumerate}

\subsection{Interpretation}

\begin{enumerate}
\item Where a dispute may arise with regards to the interpretation of this Constitution, the Dispute Resolution Committee shall in accordance with Article 12 make a determination with regards to the dispute.
\end{enumerate}

\subsection{Power to Make By-Laws}

\begin{enumerate}
\item The National Council:
\begin{enumerate}
\item Has authority to enact by-laws that, within the constraints of this Constitution, may affect or clarify this Constitution;
\item Is empowered with authority to enact, amend or revoke by-laws; and
\item Must keep a register of all such by-laws which shall be available to members on request.
\end{enumerate}
\end{enumerate}

\subsection{Operational and Temporary Amendments}

\begin{enumerate}
\item A three-quarters majority of the National Council is empowered to make alterations to the Constitution where circumstances of urgency dictate, or where it is necessary for party operation.
\item Such alterations are temporary, and are considered proposed amendments and as such must be voted upon by Members at the next National Congress, where (if approved) they shall become amendments; and
\item If such an amendment does not receive the necessary majority as stipulated at Article 9.1, then such a proposed amendment will lapse and the National Council may not use their powers to resurrect the provisions again without an amendment proposal being put to the National Congress.
\end{enumerate}

\section{Councillors, Officers and Party Officials}

\subsection{Election}

\begin{enumerate}
\item The positions enumerated within Article 3.3 will be appointed by election through a vote of the Full Members at the National Congress, for a term that shall begin at the National Congress at which they are elected, and will all end at the next Annual National Congress, except where otherwise provided for in this Constitution.
\item The members who are elected to positions on the National Council at the National Congress will take up those positions seven (7) days after the result is announced.
\item The outgoing members of the National Council must hand over and communicate as much relevant knowledge as is feasible.
\item Those members that nominate themselves, or are nominated, for a position on the National Council, working group or committee must consent in writing to their nomination.
\item No more than one National Council position may be filled by one member, except in cases where a position is subject to a temporary vacancy and pending a permanent appointment. In any case, no member of the National Council may cast more than one vote in any motion before the National Council.
\item A member of the National Council is automatically considered to be unable or unwilling to perform their duties if they fail to attend a meeting four (4) consecutive times or for two (2) consecutive months, whichever occurs first, and the position is automatically declared vacant.
\item The National Council may declare a National Council member unable or unwilling to perform their duties and render that position vacant by an absolute two-thirds majority vote of those remaining members of the National Council.
\item In the event a vacancy appears on the National Council:
\begin{enumerate}
\item If at the last National Congress there were further candidates for the vacant position, the National Council must invite the next strongest candidate for that position onto the National Council, and repeat until the list exhausts. The National Council has seven (7) days to fill the vacancy using this method.
\item If there are no candidates or the time expires, the position is announced as vacant and an election is called for twenty-one (21) days from the date of the expired timer or date of resignation if no timer was triggered.
\item Other than the above, the election procedure is to follow the voting procedures of a National Congress, and may be held entirely online.
\item If no candidates stand for election, the National Council may opt to appoint a member to the National Council by absolute two-thirds majority vote of the remaining members of the National Council.
\end{enumerate}
\item The voting method to be used for elections at the National Congress shall be the optional preferential Schulze method.
\item Each member is only entitled to vote once in each election.
\item Candidates for any electable position or appointment within the Party must present a declaration of any potential conflicts of interest prior to the election or appointment taking place.
\end{enumerate}

\section{Constitution Not Enforceable in Law}

\begin{enumerate}
\item In this Article, Constitution means all constituent documents of the Party, all resolutions of the National Congress and all resolutions of the National Council relating to the structure and organisation of the Party.
\item It is expressly intended that all disputes within the Party, or between one member and another that relate to the Party will be resolved in accordance with the Constitution and not through legal proceedings.
\item By joining the Party and remaining members, all members of the Party consent to be bound by this Article.
\end{enumerate}

\section{Dispute Resolution Committee}

\begin{enumerate}
\item At the National Congress, members of the Party must elect three (3) members who will form the Dispute Resolution Committee, as per the requirements of Article 10.1.
\begin{enumerate}
    \item If any committee members, elected at a previous Congress, have not completed serving their term by the end of the congress, their position will not be considered up for reelection, and the number of members to be elected will be reduced accordingly.
\end{enumerate}
\item The tenure of a Dispute Resolution Committee member will be two (2) National Congresses and also includes any intervening emergency National Congress.
\item A member of the Dispute Resolution Committee may only be removed from that position if that Member chooses to resign from the membership of the Party, or through a three-quarters majority of the National Council.
\item Members will first attempt to resolve disputes in good faith amongst themselves, then may appeal to the Deputy President for assistance in resolving the dispute. Only after attempts to resolve the dispute have failed, then those members may apply to the Dispute Resolution Committee to resolve that dispute.
\item The Dispute Resolution Committee will only hear an appeal after all feasible avenues of appeal or resolution, in accordance with any policy or by-law, or the policy or by-laws of the subordinate organisation or other organ of the Party have been exhausted.
\begin{enumerate}
  \item This section does not apply if a policy or bylaw that would prevent referral to the DRC is the policy or bylaw being disputed.
\end{enumerate}
\item The Dispute Resolution Committee will be responsible for hearing all appeals for resolution of disputes between members, where that dispute:
\begin{enumerate}
\item is in relation to the Party;
\item relates to compliance with, or interpretation of, this Constitution; or
\item relates to any rights or obligations of members, subordinate organisation or organ of the party.
\end{enumerate}
\item The Dispute Resolution Committee must:
\begin{enumerate}
\item comply with rules of procedural fairness;
\item conduct its proceedings as expeditiously as is possible;
\item refer its determination in writing to the National Council.
\item declare any real or apparent conflicts of interest its members may have in relation to the dispute prior to the dispute being heard.
\end{enumerate}
\item The National Council must:
\begin{enumerate}
\item as expeditiously as possible implement the determinations of the Dispute Resolution Committee;
\item publish and make available to members the result of such a determination.
\end{enumerate}
\end{enumerate}

\section{Dissolution}

\begin{enumerate}
\item The Party may only be dissolved by a postal ballot - where more than two-thirds of members elect to dissolve, and not less than half of current Full Members participate in that ballot. Members will be given at least three months advance notice of this proposal to disband.
\item Dissolution is effective within 30 days of the results of the ballot being formally announced, or whatever date that ballot may specify.
\item If, after the election to dissolve, all liabilities and debts have been satisfied, and remaining costs and fees with regards to the dissolution have been accounted for, there remains property belonging to the Party, that remainder shall be distributed to any organisation with similar goals and principles as set out in this Constitution.
\end{enumerate}

\end{document}
