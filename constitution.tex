\documentclass[a4paper,titlepage,8.5pt]{article}
\usepackage{fontspec}
\setmainfont[Ligatures=TeX]{DejaVu Sans}
\usepackage[english]{babel}
\usepackage{csquotes}
\usepackage[titles]{tocloft}
\usepackage{hyperref}
\usepackage[parfill]{parskip}
\usepackage{a4wide}
\usepackage{draftwatermark}

\title{Pirate Party Australia Constitution}

\hypersetup{
	pdftitle={Pirate Party Australia Constitution},
	pdfauthor={},
	pdfsubject={},
	pdfcreator={},
	pdfproducer={},
	pdfkeywords={},
	colorlinks=false % set to true to see links in red
}
% TOC configuration
%\def\thepart{Article \Roman{section}}
%\def\thesection{Article \arabic{section}}

%\def\thesubsection{- \Roman{section}\alph{subsection}}
%\renewcommand{\cftsubsecdotsep}{\cftnodots}

% Set section indent
\setlength{\cftsecnumwidth}{2.0em}
%\setlength{\cftsecindent}{2.0em}

% Set subsection indent
%\setlength{\cftsubsecnumwidth}{5.0em}
%\setlength{\cftsubsecindent}{2.0em}

% Remove part prefix in actual document content
%\def\partname{}

\begin{document}

\maketitle
\thispagestyle{empty}

\tableofcontents
\thispagestyle{empty}
\newpage


% Reset page numbers so the TOC doesnt count...
\setcounter{page}{1}

\part{Principles \& Objects of the Party}
Pirate Party Australia strives to achieve the freedom of culture, the protection of civil liberties and inalienable rights of the nation’s citizens and to protect the freedoms of the newly evolving global information society. It is these values that the party seeks to have embedded within the laws and institutions of Australia.

The growing surveillance of the citizen offends the very notions of a liberal and open democracy. Overbearing and restrictive private monopolies constructed via regimes of antiquated, unfair and unbalanced laws which prevent the free development of culture and ideas are detrimental to financial, economic and cultural outcomes for the citizens of Australia. Changing these laws, and ensuring the protection of these values are the goals of the Pirate Party Australia. 

Founded on the same principles as other International Pirate Parties, it is part of a global movement against increasingly draconian copyright and patent laws, and the erosion of the right to privacy. The basic tenets of this movement are Free Culture, Civil Liberty and Intellectual Rights Reform.

The Party does not seek to become part of the administration, but a mediator in parliamentary deadlocks, and to provide representation for the emergent information society, to guard the civil liberties of the citizen by utilising this power to further the party agenda, and as such intends to contest Australian Federal Elections in both the House of Representatives and Senate.

An elected representative of the party must not vote for or compromise on any legislation that impinges on or compromises the rights stated here in this constitution.


Objectives of the Pirate Party Australia also include:
\begin{itemize}
\item To construct, advocate and implement policies in accordance with the principles stated within this constitution; and
\item To generally educate and bring awareness to the issues that are stated within this constitution; and
\item To educate and encourage other political entities to adopt our objectives, whether that be through advocacy or preference allocation.
\end{itemize}


The Pirate Party Australia firmly holds belief in democracy, and rejects any use of force, intimidation or physical violence as the means to achieving political goals. We vehemently reject any and all forms of political or public corruption.
\newpage

\part{Articles of the Constitution}

\section{Name and Principles and Constitution}

The name of the party will be ``Pirate Party Australia'', also known as ``The Pirate Party of Australia'', or ``Pirate Party'', and from hereinafter in this document shall be referred to as the ``PPAU''. The principles and objects of the PPAU are stated in Part I, and are fundamental to the purpose of the party. All party documents, members and policies are subject and subordinate to this constitution. 
\section{Structure \& Composition}

\begin{enumerate}
\item The Party shall be governed at a Federal level by a body entitled the ``National Council''. The National Council may create additional structures, such as a National Executive, Parliamentary Party, Women Pirates Committee and Young Pirates Party or such other structures as it sees fit.
\item The National Council shall be comprised of those persons formally elected to positions elaborated on within this constitution.
\item Those members who form the National Council are to be elected from those eligible persons as elaborated within this constitution, and are to be elected after deliberation at an annual National Congress, for a period of no more than twelve months.
\item The elected members of the National Council shall appoint to the National Council a Party Agent in accordance with Article 8. The Party Agent will fulfil the obligations and duties of the role as provided by the Commonwealth Electoral Act 1918.
\item The National Council, as the paramount governing body of the PPAU, has authority to overrule or amend policy of any subordinate organisation (except those policies of the Dispute Resolution Committee), or decisions of that subordinate organisation (except those determinations made by the Dispute Resolution Committee), if it deems them to be inconsistent with, or it perceives that policy or organisation to be repugnant to the policy of the National Council’s policies, or the values and ideals of the PPAU. A two thirds majority vote of the National Council is required for any such action to be taken against a subordinate organisation or their decisions.
\end{enumerate}

\section{Membership}

\subsection{Full Membership}

\subsubsection{Full Membership is open to:}

\begin{enumerate}
\item All natural persons who;
\begin{enumerate}
\item Have read and agreed to the terms and principles provided within this constitution;
\item Pay an annual membership fee, if applicable, as set by the National Council;
\item Are not currently members of any other registered political party and do not join another party whilst a member of the PPAU; and
\item Have not been members of another registered political party in the previous twelve (12) months unless permission is granted in writing (in their absolute discretion) by a majority of the National Council.
\end{enumerate}
\end{enumerate}

\subsubsection{All Full Members are entitled to:}

\begin{enumerate}
\item Be elected into a formal position within the party, at any level;
\item Where eligible, and approved by the nomination processes within this constitution, stand as a candidate in any election the party contests;
\item Communicate and submit policy amendment proposals and constitutional amendment proposals;
\item Participate in policy and issue discussion, debate and partake in the decision making process in accordance with this constitution;
\item Participate in working groups defined by the National Council or any organ of the Party; and
\item Vote at Party Meetings, Congresses and Policy Formulation, Development and Adoption proceedings.
\item A Member's Party membership will only lapse if the member fails to pay their annual membership fee for more than two consecutive years. The National Council will give thirty days written notice to the last known postal or email address of the member in advance of declaring that membership to have lapsed. Such a lapsed membership may be re-activated during its third year of lapse by that Member paying out of their own pocket their outstanding membership fees. A member may notify the National Council that they wish to resign their membership of the Party by giving sixty (60) days
signed written notice.
\end{enumerate}

\subsection{Additional Categories of Membership:} 

A majority vote of the National Council may propose to existing members the creation of additional categories of Membership of the Party. A quorum of 10\% of the entire existing membership is necessary for such a vote, which will be decided by simple majority of those members. 

\subsection{Recruitment of Members}
The National Council may specify rules to prevent the abusive recruitment of new members into the Party or abusive renewal of memberships. 

\subsection{Founding Members}
\begin{enumerate}
\item Founding Members will pay a fee of \$20.00 to the party;
\begin{enumerate}
\item within sixty (60) days of application;
\item that will be offset against any applicable membership dues set in the first year by the National Council.
\end{enumerate}
\end{enumerate}

\subsection{National Council Powers}

\subsubsection{Refusal, Suspension and Expulsion}

\begin{enumerate}
\item The National Council may refuse to accept an application for membership by any individual on the grounds that the acceptance of the membership may be prejudicial to the interests or values of the PPAU.
\item The National Council also has the power to suspend or expel a member should that individual’s membership or actions whilst a member be prejudicial to the interests of the PPAU.
\item Any refusal to admit a person as a member, and any suspension or expulsion of a member, shall be accompanied by a written statement of reasons for the action, and this statement is to be made available to all membership (unless requested to be kept confidential by the member affected).
\item A refusal of an application for membership, or the suspension or expulsion of any member may only be decided by a two thirds majority vote of the National Council.
\end{enumerate}

\section{Policy Formulation, Development and Adoption}

\subsection{Development}

Policy development should occur with the maximum possible interaction with the party members - the party should engage in as a participatory process as is possible, and outcomes should be reached through consensus.

\subsection{Adoption}

\begin{enumerate}
\item New policy, unless dictated by circumstances of urgency, shall be decided on at the National Congress.
\item New policy should be agreed upon by a general consensus. However, if a majority of the National Council votes that there is an the absence of a general consensus, a two-thirds majority of all votes cast by the members must be in the affirmative for new policy to be adopted. 
\item Where circumstance of urgency are apparent and declared, the National Council may make policy, that shall be considered official, however that policy is subject to vote at the next National Congress, and is subject to the same conditions as those above. 
\item A policy will not be adopted if it is inconsistent with Part I of this constitution.
\item All policies adopted by the National Congress will be recorded in a central register available to all Members.
\end{enumerate}

\subsection{Policy Review}

\begin{enumerate}
\item Where not less than 15\% of Full Members petition the National Council, a policy will come under official review by the party, where that policy will be reviewed and voted upon at the National Congress.
\end{enumerate}

\section{Meeting Procedure and Requirements}

\begin{enumerate}
\item Meetings should be structured so as to allow all members to participate, and have their opinions acknowledged.
\item All members should be notified at least 24 hours in advance of any official meeting of the National Council and at least seven days in advance of the National Congress, and of the intended agenda of such meetings.
\item Consensus should be the focus of any proposal or decision at a meeting- however where consensus cannot be achieved, a two-thirds majority will be sufficient to carry forward a proposal.
\item Where there is disagreement, or members indicate that a delay in voting is required, sufficient time should be given for discussion before any voting begins.
\item Meetings are only open to members unless a majority of the members present permit specified non-members to observe the meeting.  
\item The method of voting is to be determined by the meeting facilitator and the medium by which the meeting occurs.
\item The minutes of a meeting should be distributed to the Members within seven days of the meeting. The National Council may specify procedures for the collection and dissemination of such minutes.
\item The National Council may specify additional meetings procedures. 
\end{enumerate}

\subsection{National Congress}

\begin{enumerate}
\item The National Council will organise the National Congress.
\item The PPAU will hold a National Congress within 12 months of its previous National Congress.
\item If at least 25\% of the members petition the National Council in writing expressing their lack of confidence in the National Council, the National Council shall organise an emergency National Congress of the Members within thirty (30) days. 
\end{enumerate}

\subsection{Pre-Selection of Candidates for Election to Federal Parliament}

\begin{enumerate}
\item All Members seeking to stand as candidates for election to Federal Parliament must be nominated at the National Congress and seconded by another member.
\item The National Council will determine whether all members (or a geographical sub-set of members) will vote to select candidates for election to Federal Parliament.
\item All members seeking to stand as candidates must submit to the National Congress a detailed and truthful statement as to their suitability for election.
\item The National Council may establish procedures for the vetting of candidates backgrounds and must publish these procedures to the Membership. 
\item As far as is practicable, candidates should be selected at least twelve (12) months before the normal time of the next election.
\item All members wishing to run as candidates for Pirate Party Australia must sign a declaration to the effect of:
\begin{enumerate}
	\item I hereby pledge to advance and adhere to the platform and ideals of Pirate Party Australia, both during the election campaign and upon election to Parliament.
\end{enumerate}
\end{enumerate}

\section{Financial Structure}

\subsection{Property}

\begin{enumerate}
\item All property and resources of the PPAU are to be used solely for the purposes of promoting and achieving the principles and goals stated within this constitution.
\item All Members, upon request to the National Council, may have access to the latest financial reports of the PPAU.
\item All bank accounts of the PPAU will:
\begin{enumerate}
\item be held separately from those of its members;
\item require more than one signatory for the disbursement of funds; and 
\item include the wording PPAU in their title.
\end{enumerate}
\item All accounts of the PPAU will be audited annually and the auditor’s report published to the members at the National Congress.
\end{enumerate}

\section{Constitutional Amendments, Interpretation and By-Laws}

\subsection{Amendments}

\begin{enumerate}
\item The constitution may only be amended during the National Congress. Amendments require a two-thirds majority vote with a quorum of fourteen percent of Members at the time the amendment was proposed.
\item Any proposals for amendments must be notified in writing to the members at least 28 days prior to the National Congress.
\item At each subsequent National Congress, the members will vote on whether to raise the quorum in Article 7, clause 1 by an additional two (2) percent (eg from 10\% to 12\% to 14\%, etc). In the event that the Members do not vote in favour of that increase, then this clause will lapse.
\end{enumerate}

\subsection{Interpretation}

\begin{enumerate}
\item Where a dispute may arise with regards to the interpretation of this constitution, the Dispute Resolution Committee shall in accordance with Article 10 make a determination with regards to the dispute.
\end{enumerate}

\subsection{Power to Make By-Laws}

\begin{enumerate}
\item The National Council:
\begin{enumerate}
\item Has authority to enact by-laws that, within the constraints of this constitution, may affect or clarify this constitution;
\item Is empowered with authority to enact, amend or revoke by-laws; and
\item Must keep a register of all such by-laws which shall be available to members on request.
\end{enumerate}
\end{enumerate}

\subsection{Operational and Temporary Amendments}

\begin{enumerate}
\item A three-quarters majority of the National Council is empowered to make alterations to the constitution where circumstances of urgency dictate, or where it is necessary for party operation.
\item Such alterations are temporary, and are considered proposed amendments and as such must be voted upon by Members at the next National Congress, where (if approved) they shall become amendments; and
\item If such an amendment does not receive the necessary majority as stipulated at Article 7 (1), then such a proposed amendment will lapse and may only be resurrected by a majority vote of the members at a National Congress.
\end{enumerate}

\section{Officers and Party Officials}

\subsection{Election}

\begin{enumerate}
\item The positions stated within this constitution will be appointed by election through a vote by the Members at the National Congress, for a term not exceeding twelve (12) months, and shall form the National Council. Their term shall begin at the National Congress at which they are elected and end at the next National Congress.
\item Those members that nominate themselves, or are nominated, for a position on the National Council must consent in writing to their nomination.
\item No more than two officer positions may be filled by one member of the National Council.
\item In the event a member of the National Council is unable or unwilling to perform their duties, the remaining members of the National Council may declare the position vacant and appoint an interim replacement by two-thirds majority vote of the remaining members of the National Council. The next National Congress will then elect a Member to fill that vacant position.
\item The method for voting by members at the National Congress will be optional preferential voting, with the method for the next Congress determined by motion at the National Congress. The voting method carries until the next National Congress if a two-thirds majority of members present at the Congress is not met for an alternatively proposed method.
\item Each member is only entitled vote once in each election.
\end{enumerate}

\subsection{President}

\subsubsection{Duties and Responsibilities}

\begin{enumerate}
\item Lead the PPAU.
\item Chair the National Congress, and meetings of the National Council.
\item Co-ordinate the activities of the National Council.
\end{enumerate}

\subsection{Deputy President}

\subsubsection{Duties and Responsibilities}

\begin{enumerate}
\item Assist the President with their duties in accordance with this constitution.
\item If the President is unable (on a temporary basis) to conduct their obligations under the constitution, the Deputy is to substitute and fulfil those obligations.
\end{enumerate}

\subsection{Party Secretary}

\subsubsection{Duties and Responsibilities}

\begin{enumerate}
\item Provide notice in advance to members of all official meetings, and of the National Congress.
\item Prepare schedules, agenda, and correspondence from members for submission to the meeting or National Congress, and record attendance of persons present, and arrange for minutes or logs to be recorded.
\item Co-ordinate official correspondence of the National Council.
\item Maintain the party register, in accordance with Commonwealth Electoral Act 1918.
\item Maintain custody of all documents, statements and records of the PPAU, and except for those documents that are otherwise accounted for in this constitution, by other officers.
\item Briefly minute, or delegate responsibility for minuting, listing the decisions of meetings of the National Congress and National Council and ensure publication at the earliest possible convenience.
\end{enumerate}

\subsection{Deputy Party Secretary}

\subsubsection{Duties and Responsibilities}

\begin{enumerate}
\item Assist the Party Secretary with their duties in accordance with this constitution.
\item If the Party Secretary is unable (on a temporary basis) to conduct their obligations under the constitution, the Deputy is to substitute and fulfil those obligations.
\end{enumerate}

\subsection{Treasurer}

\subsubsection{Duties and Responsibilities}

\begin{enumerate}
\item The receipt of all monies paid to the PPAU, the issuing of all receipts and the deposit of such monies into accounts determined by the National Council.
\item Develop and ensure security and accountability measures for all receipts and payments are followed.
\item Submit an Annual Financial Report to the National Congress, detailing balance sheets, financial statements and relevant particulars.
\item Maintain adequate controls over PPAU finances and all financial records, documents, securities ensuring smooth transition when position is transferred.
\item Ensure that all book keeping is conducted by an appropriately skilled person, and all documents conform to relevant legislation and regulations and this Constitution.
\end{enumerate}

\subsection{Deputy Treasurer}

\subsubsection{Duties and Responsibilities}

\begin{enumerate}
\item Assist the Treasurer with their duties in accordance with this constitution.
\item If the Treasurer is unable (on a temporary basis) to conduct their obligations under the constitution, the Deputy is to substitute and fulfil those obligations.
\end{enumerate}

\subsection{Party Agent}

\subsubsection{Duties and Responsibilities}
\begin{enumerate}
\item A member elected by the National Council in accordance with this constitution will act as Party Agent, so as to fulfil the requirements and obligations of the Commonwealth Electoral Act 1918.
\item At the National Congress, up to five candidates for the position of Party Agent may be nominated and voted upon by the members. Each candidate that attracts at least 10\% of the votes cast for Party Agent by the Members will then be presented to the National Council for consideration. The National Council will then vote (by 2/3rd majority) to select one of those presented candidates as the Party Agent.
\end{enumerate}

\section{Constitution Not Enforceable in Law}

\begin{enumerate}
\item In this Article, Constitution means all constituent documents of the Party, all resolutions of the National Congress and all resolutions of the National Council relating to the structure and organisation of the Party.
\item It is intended that the Constitution and everything done in connection with it, all arrangements relating to it (whether express or implied) and any agreement or business entered into or payment made under the National Constitution will not bring about any legal relationship, rights, duties or outcome of any kind, or be enforceable by law, or be the subject of legal proceedings. Instead, all arrangements, agreements and business are only binding in Honour.
\item Without limiting Article 9.2, it is further expressly intended that all disputes within the Party, or between one member and another that relate to the Party will be resolved in accordance with the Constitution and not through legal proceedings.
\item By joining the Party and remaining members, all members of the Party consent to be bound by this Article.
\end{enumerate}

\section{Dispute Resolution Committee}
 
\begin{enumerate}
\item At the National Congress, members of the Party must elect by simple majority elect three (3) members who will form the Dispute Resolution Committee.
\begin{enumerate}
\item At least one member of the Dispute Resolution Committee should be legally qualified.
\end{enumerate}
\item The tenure of a Dispute Resolution Committee member will be two (2) National Congresses and also includes any intervening emergency National Congress.
\item A member of the Dispute Resolution Committee may only be removed from that position if that Member chooses to resign from the membership of the Party, or through a three-quarters majority of the National Council.
\item Members will first attempt to resolve disputes in good faith amongst themselves, then may appeal to the Deputy President for assistance in resolving the dispute. Only after attempts to resolve the dispute have failed, then those members may apply to the Dispute Resolution Committee to resolve that dispute.
\item The Dispute Resolution Committee will only hear an appeal after all feasible avenues of appeal or resolution, in accordance with any policy or by-law, or the policy or by-laws of the subordinate organisation or other organ of the Party have been exhausted.
\item The Dispute Resolution Committee will be responsible for hearing all appeals for resolution of disputes between members, where that dispute:
\begin{enumerate}
\item is in relation to the Party;
\item relates to compliance with, or interpretation of, this constitution; or
\item relates to any rights or obligations of members, subordinate organisation or organ of the party.
\end{enumerate}
\item The Dispute Resolution Committee must:
\begin{enumerate}
\item comply with rules of procedural fairness;
\item conduct its proceedings as expeditiously as is possible;
\item refer its determination in writing to the National Council.
\item declare any real or apparent conflicts of interest its members may have in relation to the dispute prior to the dispute being heard.
\end{enumerate}
\item The National Council must:
\begin{enumerate}
\item as expeditiously as possible implement the determinations of the Dispute Resolution Committee;
\item publish and make available to members the result of such a determination.
\end{enumerate}
\end{enumerate}

\section{Dissolution}

\begin{enumerate}
\item The PPAU may only be dissolved by a postal ballot - where more than two thirds of members elect to dissolve, and not less than half of current Full Members participate in that ballot. Members will be given at least three months advance notice of this proposal to disband.
\item Dissolution is effective within 30 days of the results of the ballot being formally announced, or whatever date that ballot may specify.
\item If, after the election to dissolve, all liabilities and debts have been satisfied, and remaining costs and fees with regards to the dissolution have been accounted for, there remains property belonging to the PPAU, that remainder shall be distributed to any organisation with similar goals and principles as set out in this constitution.
\end{enumerate}

\end{document}
